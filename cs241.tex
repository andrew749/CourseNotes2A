\documentclass[11pt]{amsart}
\usepackage{geometry}
\geometry{letterpaper}
\usepackage{graphicx}
\usepackage{amssymb}
\usepackage{epstopdf}
\DeclareGraphicsRule{.tif}{png}{.png}{`convert #1 `dirname #1`/`basename #1 .tif`.png}

\title{CS241}
\author{Andrew Codispoti}

\begin{document}
\maketitle
\section{Binary and Hexadecimal numbers}
\begin{enumerate}
  \item bit -- binary digits 1 and 0 (all computer understands)
  \item byte -- 8 bits
  \item word
    \begin{enumerate}
      \item machine specific grouping of bits
      \item assume 32-bit architecture
      \item 1 word = 32 bits= 4 bytes
    \end{enumerate}

  \item nibble -- 4 bits half a byte
\end{enumerate}


\subsection{Given a byte(or word) in memory what does it mean?}
\par Could mean many things.
\begin{enumerate}
  \item A number (which number?)
\end{enumerate}
\subsection{How can we represent negative numbers?}
\par Simply use a sign bit with 0 for + and 1 for - (Sign-Magnitude
representation) but then you have two -1's and arithmetic is tricky
\subsubsection{Two's Complement notation}
\par Interpret the n-bit number as a an unsigned int. If first bit is 0 done else
subtract $2^n$
\par n bits- represent $-2^{n-1}$ \dots $2^{n-1}$ with left bit still giving
sign. arithmetic is clean, just mod $2^n$
\par We cant tell if a number is signed unsigned or two's complement and we
have to remember.
\par We don't even know if what it means:a number, a character, An
instruction (or part of one), Garbage
\subsection{Hexadecimal notation}
\begin{enumerate}
  \item base 16 0-9, A-F
  \item more compact than binary
  \item each hex digit = 4 bits (1 nibble)
  \item e.g. 1100 1001  = C9
  \item NOTATION: 0xC9
\end{enumerate}

\subsection{Mapping from binary to characters}
\subsubsection {ASCII}
\par Uses 7 bits
\par IBM implemented extended ascii to use all 8-bits, but they add some weird
characters i.e. frame like characters.Compatibility issues because no one
standard.
\par 11001001 is not 7 bit ascii, 01001001 decimal 73 is ASCII for I
\par other standards like EBCDIC
\section{Machine Language}
\par Computer programs operate on data and are data(occupy same space as data)
\subsection{Von Neumann architecture}
\par Programs reside in the same memory as data.
\par Programs can operate on other programs i.e OS
\end{document}
