\documentclass[11pt]{amsart}
\usepackage{geometry}
\geometry{letterpaper}
\usepackage{graphicx}
\usepackage{amssymb}
\usepackage{epstopdf}
\usepackage[T1]{fontenc}
\DeclareGraphicsRule{.tif}{png}{.png}{`convert #1 `dirname #1`/`basename #1 .tif`.png}

\title{CS241 Tutorial}
\author{Andrew Codispoti}

\begin{document}
\maketitle
\section{STL}
\subsection{containers}
\begin{enumerate}
    \item pair<T,U>: group two items
    \item vector<T>: array like sequence of elements of type T
    \item list<T>: linear sequence of elements of type T
    \item map<T,U>: dictionary with keys of type T and values of type U
    \item set<T>: collection of elements of type T with fast membership lookup
\end{enumerate}
\subsection{Algorithms}
\begin{enumerate}
    \item find: get iterator to element
    \item count: count number of elements that compare equal
    \item copy: copy elements of range to another range
    \item for\_each: apply function to each element
    \item transform: like map
    \item accumulate: adds all values in parameters plus an init
\end{enumerate}

\section{Tutorial 2}
\subsection{Example 1}
\par find fibonacci number and place into \$3
\par refer to fibonacci.asm for code
\subsection{Example 2}
\par convert problem 1 into a procedure  named fib  which expects \$1 to be n
and put result in \$3. every register should have same value after program
returns.
\section{Tutorial 3}
\subsection{symbol table}
\begin{tabular}{|c c|}
  \hline
  name & value \\
  \hline
  begin & 0\\
  label & 0\\
  after & 8\\
  abc0 & 0x10\\
  abc1 & 0x10\\
  loadstore &0x14\\
  end & 0x1c\\
  \hline
\end{tabular}
\subsection{error identification}
\begin{enumerate}
  \item repeat label names
  \item no operant for .word
  \item label does not exist
  \item multiple operands
\end{enumerate}
\end{document}
