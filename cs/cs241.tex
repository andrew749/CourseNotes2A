\documentclass[11pt]{amsart}
\usepackage{geometry}
\geometry{letterpaper}
\usepackage{graphicx}
\usepackage{amssymb}
\usepackage{epstopdf}
\DeclareGraphicsRule{.tif}{png}{.png}{`convert #1 `dirname #1`/`basename #1 .tif`.png}
\usepackage[T1]{fontenc}
\title{CS241}
\author{Andrew Codispoti}

\begin{document}
\maketitle
\section{Binary and Hexadecimal numbers}
\begin{enumerate}
  \item bit -- binary digits 1 and 0 (all computer understands)
  \item byte -- 8 bits
  \item word
    \begin{enumerate}
      \item machine specific grouping of bits
      \item assume 32-bit architecture
      \item 1 word = 32 bits= 4 bytes
    \end{enumerate}

  \item nibble -- 4 bits half a byte
\end{enumerate}


\subsection{Given a byte(or word) in memory what does it mean?}
\par Could mean many things.
\begin{enumerate}
  \item A number (which number?)
\end{enumerate}
\subsection{How can we represent negative numbers?}
\par Simply use a sign bit with 0 for + and 1 for - (Sign-Magnitude
representation) but then you have two -1's and arithmetic is tricky
\subsubsection{Two's Complement notation}
\par Interpret the n-bit number as a an unsigned int. If first bit is 0 done else
subtract $2^n$
\par n bits- represent $-2^{n-1}$ \dots $2^{n-1}$ with left bit still giving
sign. arithmetic is clean, just mod $2^n$
\par We cant tell if a number is signed unsigned or two's complement and we
have to remember.
\par We don't even know if what it means:a number, a character, An
instruction (or part of one), Garbage
\subsection{Hexadecimal notation}
\begin{enumerate}
  \item base 16 0-9, A-F
  \item more compact than binary
  \item each hex digit = 4 bits (1 nibble)
  \item e.g. 1100 1001  = C9
  \item NOTATION: 0xC9
\end{enumerate}

\subsection{Mapping from binary to characters}
\subsubsection {ASCII}
\par Uses 7 bits
\par IBM implemented extended ascii to use all 8-bits, but they add some weird
characters i.e. frame like characters.Compatibility issues because no one
standard.
\par 11001001 is not 7 bit ascii, 01001001 decimal 73 is ASCII for I
\par other standards like EBCDIC
\section{Machine Language}
\par Computer programs operate on data and are data(occupy same space as data)
\subsection{Von Neumann architecture}
\par Programs reside in the same memory as data.
\par Programs can operate on other programs i.e OS
\subsection{Central Processing Unit}
\par see physical notes for diagram
\begin{enumerate}
  \item Control Unit
    \begin{enumerate}
      \item decodes instructions
      \item dispatches to other parts of the computer to carry out instructions
    \end{enumerate}
  \item Arithmetic logic unit: Does Math
\end{enumerate}
\subsection{Memory--Many Kinds (Ranked in speed order)}
\begin{enumerate}
  \item **CPU
  \item cache
  \item **main memory RAM
  \item disk memory
  \item network memory
\end{enumerate}
\subsection{Registers}
\par On the CPU, small amount of very fast memory called registers
\par MIPS 32 General purpose registers \$0 to \$31
\begin{enumerate}
  \item each holds 32 bits
  \item can only operate on data that is in regs.
  \item \$0 is always 0
  \item \$31 is special and \$30
  \item EX: add the contents of two registers and put the result in another
    register.
  \item 5 bits encode a register $2^5 = 32$
  \item 15 bits to encode registers, 17 bits to encode operation
\end{enumerate}
\subsection{RAM}
\begin{enumerate}
  \item large amount of memory away from cpu
  \item travels between the cpu and ram on the bus
  \item big array of n-bytes, n ~ $10^9$
  \item each cell has an address 0,\dots, n-1
  \item each 4-byte block of the form is a word(see diagram 2 in notes)
  \item word addresses are 0,4,8,c,10,14,18,1c
  \item RAM access much slower than reg access
\end{enumerate}
\subsection{Communicating with RAM}
\par two commands
\begin{enumerate}
  \item load
    \begin{enumerate}
      \item transfer a word from an address to a register. desired address
        goes into the memory address register(MAR), goes out on bus
      \item data at that location comes back on the bus, goes into the memory
        data register (MDR)
      \item value in MDR moved to destination register
    \end{enumerate}
  \item store: does the reverse of load
\end{enumerate}
\subsection{How does a computer know which words contain instructions and which
  contain data?}
\par It Doesn't
\subsection{How does it run}
\par Special register called pc(program counter) which stores the address of
the next instruction to execute
instruction to execute.
\par By convention, guarantee that some address(i.e. 0) contains code,
initialize pc to 0.
\par Computer then runs the fetch-execute cycle
\begin{verbatim}
  PC <- 0
  loop
  IR <- MEM[PC]
  PC <- 4
  decode and execute the instruction in IR
  end loop
\end{verbatim}
\par only program the machine really runs
\par \textbf{NOTE:}  PC holds the address of the next instruction while the
current instruction is executing.
\subsection{How does a program get executed}
\par Program called a loader that puts the program in memory and sets PC to teh
address of the first instruction in the program and sets PC to teh address of
the first instruction in the program
\subsection{What happens when a program ends?}
\par need to return control to the loader, set pc to the address of the next
instruction in the loader.\$31 will contain the right address.
\par need to set pc to \$31
\subsection{Example}
\par Example1: Add value in \$5 to the value in the \$7 sotre result in \$3 and
return\\
\begin{tabular}{|c c c c|}
  \hline
  location & binary & hex & meaning\\
  \hline
  00000000 & 0000 0000 1010 0111 0001 1000 0010 0000 & 00a71820 & add \$3, \$5,
  \$7 \\
  \hline
  00000004 & 0000 0011 1110 0000 0000 0000 0000 1000 & 03e00008 * jr \$31 \\
  \hline
\end{tabular}
\par Example 2 add 42 to 52, store in \$3, return
\par lis \$d  "load immediate and skip", treat the next word as an immediate
value and load it in to \$d then skip to the instruction\\
\begin{tabular}{|c c c c|}
  \hline
  location & binary & hex & meaning\\
  \hline
  00000000 & 0000 0000 0000 0000 0010 1000 0001 0100 & 00002814 & lis \$5\\
  \hline
  00000004 & 0000 0000 0000 0000 0000 0000 0010 1010 & 00000004 & .word 42 \\
  \hline
  00000008 & 0000 0000 0000 0011 1000 1000 0001 0100 & 00002814 & lis \$7\\
  \hline
  0000000c & 0000 0000 0000 0000 0000 0000 0011 0100 & 00000004 & .word 52 \\
  \hline
  00000004 & 0000 0011 1110 0000 0000 0000 0000 1000 & 03e00008 & jr \$31 \\
  \hline
\end{tabular}
\begin{verbatim}
  add \$3 \$5 \$7
\end{verbatim}
\subsection{assembly language}
\par replace tedious binary/hex encodings with easier to read mnuemonics
less chance of error, translation to binary can be automated (assembler), one
line of assembly = one machine instruction (word)
\begin{verbatim}
  lis $5; load imm and skip
  .word 42; not an instruction, is a directive that next word in binary should
  literally be 42
  list $7;
  .word 52
  add $3 $5 $7;destination reg first
  jr $31
\end{verbatim}
\subsubsection{EX3}
\par Compute the absolute value of \$1 store in \$1 and return
\begin{enumerate}
  \item some insturcitons modify PC =  "branches" and "jumps", i.e. jr
  \item beq: branch if 2 registers have equal contents, increment PC by a given
    number of words, can branch backwards.
  \item lone bne
  \item slt: "set less than"
    \begin{verbatim}
      slt $a, $b, $c
      $a = {1 : $b<$c>, 0}
    \end{verbatim}

\end{enumerate}

\subsubsection{RAM}
\par lw= load word from ram into reg
\begin{verbatim}
  lw $a, i($b) loads the word at MEM[$b + i]into $a
\end{verbatim}

\par sw = store word from regs into RAM
\begin{verbatim}
  sw $a , i($b) stores word in $a at mem[$b+1]
\end{verbatim}
\par see examples on paper.
\subsection{Multiplication}
\par mult =  multiply
\begin{verbatim}
  mult $a, $b ; Product of 2 32 bit numbers up to 64bitz, too big for register
\end{verbatim}
\par Concatenation of hi and lo registers is the entire product of
multiplication
\par mflo = move from lo
\begin{verbatim}
  mflo $a, ; $a <- lo
\end{verbatim}
\par for division lo stores quotient, and hi stores remainder
\subsection{revisit looping example}
\par have to keep track of offsets if instuctions are added or remove
\subsection{Assembler}
\par assembler allows labeled instrcutions
\begin{verbatim}
  foo: add $1, $2, $3
\end{verbatim}
\par assembler associates the name foo with the address of the instruction
\par assembler calculates the distance between the label and the program
counter, in words $\frac{top-PC}{4}$
\section{Procedures in MIPS}
\par 2 problems to solve
\begin{enumerate}
  \item call and return : transferring control into and out of the procedure (
    Procedures calling other Procedures)
  \item Registers: what if a proc overwrites my registers.
\end{enumerate}
\par We could : reserve some registers for f and some for main line so they
wont interfere. but when using recursion we run out of registers.
\par Instead guarantee that procs leave regs unchanged when done by storing in
RAM. Must stop processes from using the same ram becuase the same issue will
arise.
\par see diagram
\par Can allocate from one end of ram to the other for procs. Need to track what
ram is in use. Mips machine helps us \$30 initialized by the loader to just
past the last word of memory.
\par can use \$30 as a "bookmark" to separate used and unused RAM
\par diagram
\par RAM uses LIFO order \$30 is the stack pointer address at the top of the
stack
\subsection{Template for Procedures}
\begin{verbatim}
  f:  sw $2, -4($30) ;
      sw $3, -8($30); push registers f modifies on the stack
      lis $3
      .word 8   ; decrement 30
      sub $30, $30, $3
      ; body

  g:
      add $30, $30, $3 ; assuming $3 remains 8 increment 30
      lw $3, -8($30)
      lw $2, -4($30)

\end{verbatim}
\subsection{Call and return}
\begin{verbatim}
  main: ...
    lis $5
    .word f; address of line labelled f
    jr $5 ; jump to that line
    ;(HERE)
\end{verbatim}
\par Return: we need to set PC to the line after the jr(i.e. to HERE)
\par Solution: jalr
\subsubsection{jalr(Jump and link regiter)}
\par like jr, but sets \$31 to the address of the next instruction
\begin{verbatim}
  main: ...
    lis $5
    .word f; address of line labelled f
    jalr $5 ; jump to that line
\end{verbatim}
\par  Question: jalr overwrites \$31 so how do we get back to the loader, what if f calls
g
\par Answer: Save \$31 to the stack before the call and restore afterwards

\begin{verbatim}
main:
  lis $5
  .word f
  sw $31, -4($30)
  lis $31
  .word 4
  sub $30, $30, $31
  jalr $5
  lis $31
  .word 4
  add $30, $30, $31
  lw $31, -4($30)
  jr $31
\end{verbatim}

\subsection{Parameters and Results}
\par generally use regs (document)
\par if too many, use stack
\end{document}
