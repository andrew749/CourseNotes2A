\documentclass[11pt]{amsart}
\usepackage{geometry}                % See geometry.pdf to learn the layout options. There are lots.
\geometry{letterpaper}                   % ... or a4paper or a5paper or ...
%\geometry{landscape}                % Activate for for rotated page geometry
%\usepackage[parfill]{parskip}    % Activate to begin paragraphs with an empty line rather than an indent
\usepackage{graphicx}
\usepackage{amssymb}
\usepackage{epstopdf}
\DeclareGraphicsRule{.tif}{png}{.png}{`convert #1 `dirname #1`/`basename #1 .tif`.png}

\title{Earth121}
\author{Andrew Codispoti}

\begin{document}
\maketitle
\par ESL: Earth science literacy
\par ESLP: Earth science literacy principles Key concepts to know for layman
\section{introduction}
\subsection{james hutton}
\par theory of the earth (1788)
\par geologic change is slow large changes require large amount of time
\subsection{principle of uniformitarianism}
\par processes that shaped earth throughout geologic time are the same as those observable today
\subsection{sir william logan}
\par gsc, mapping, natural resources

\subsection{j. tuzo wilson}
\par geophysics, transform faults
\subsection{What is geology?}
\par study of earth
\subsection{The Earth system}
\begin{enumerate}
  \item lithosphere: rock surface and interior
  \item atmosphere: gaseous envelope
  \item hydrosphere: water, liquid, frozen
  \item biosphere: living matter
\end{enumerate}
\par earth powered by sun and heat from internal radioactive decay
\subsection{Rock Cycle}
\par how matter is recycled
\subsection{Earth's layered structure}
\par several different layers with varying physical attributes and chemical compositions


\end{document}
