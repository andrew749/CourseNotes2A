\documentclass[11pt]{amsart}
\usepackage{geometry}                % See geometry.pdf to learn the layout options. There are lots.
\geometry{letterpaper}                   % ... or a4paper or a5paper or ...
%\geometry{landscape}                % Activate for for rotated page geometry
%\usepackage[parfill]{parskip}    % Activate to begin paragraphs with an empty line rather than an indent
\usepackage{graphicx}
\usepackage{amssymb}
\usepackage{epstopdf}
\DeclareGraphicsRule{.tif}{png}{.png}{`convert #1 `dirname #1`/`basename #1 .tif`.png}

\title{Earth121}
\author{Andrew Codispoti}

\begin{document}
\maketitle
\par ESL: Earth science literacy
\par ESLP: Earth science literacy principles Key concepts to know for layman
\section{introduction}
\subsection{james hutton}
\par theory of the earth (1788)
\par geologic change is slow large changes require large amount of time
\subsection{principle of uniformitarianism}
\par processes that shaped earth throughout geologic time are the same as those observable today
\subsection{Sir William Logan}
\begin{enumerate}
  \item map coal in ontario
\end{enumerate}
\par gsc, mapping, natural resources
\subsection{J\. Tuzo Wilson}
\par geophysics, transform faults
\subsection{What is geology?}
\begin{enumerate}
  \item study of earth
  \item Earth in greek
  \item \textbf{Physical} material composition of earth
  \item \textbf{historical} development of earth through time(both slow and
    fast)
  \item earth always changing at all levels
  \item outcrops: bedrock on surface {easy for geologists to study}
  \item earth is has many hazardsj
  \item formation and occurence of vital resources
  \item maintaining supply and environmental impact of their extraction and use
\end{enumerate}
\subsection{Catastrophism}
\par earth's landscapes had been shapen by catastrophes. Attempted to explain
shape of earth in fixed time interval that it was a few thousand years old.
4004BCE
\subsection{Uniformitarianism}
\par physical, chemical, and biological laws that operate today always have
been at work
\par wearthering shrunk moutains

\par the present is key to the past
\subsection{The Earth system}
\begin{enumerate}
  \item lithosphere: rock surface and interior
  \item atmosphere: gaseous envelope,weather and climate
  \item hydrosphere: water, liquid, frozen
  \item biosphere: living matter
\end{enumerate}
\par earth powered by sun and heat from internal radioactive decay
\subsection{Rock Cycle}
\par How matter is recycled
\subsection{Earth's layered structure}
\item several different layers with varying physical attributes and chemical compositions
  \subsection{Geologic time}
  \par after radioactivity discovered, can accurately date stuff

  \subsubsection{Relative Dating}
  \par events placed in proper sequence by law of superposition
  \par law of superposition: older stuff on lower layers of sedimentary rock.
  younger on top
  \par principle of fossil sucession: fossil organisms success each other in
  definite order. time span can be determined by fossil content. could
  determine rocks of similar age based on types of fossils
  \subsubsection{Magnitude}
  \par earth 4.6 billion years old so time scale is extermely large
  \par humans a speck in time, long period without life on earth or simple
  organisms
  \subsection{Early evolution of earth}
  \par earth product of remnant from big band
  \par everything H and He
  \par nebular theory:bodies evolved from rotating cloud called solar nebula,
  gas
  \par eventually collapsed and period of contraction where things combine
  including elements
  \par lots of impact on early earth meaning hot, heavy metal sank to bottom
  \par melting formed buoyant masses for early crust
  \par iron core, primitive crust, mantle
  \subsection{Plate Tectonics}
  \par continents fit
  \par continental drift: continents moved across face of planet
  \par paradigm evolved over long period of time
  \subsubsection{continental drift}
  \par wegener ignored first to propose idea
  \par PAngea 200 mill yrs ago
  \par just being able to fit continents wasnt enought evidence
  \par continental shelf:gently sloping shelf platform of continental
  material extending from shore. fit better with this
  \par streams deposit sediment enlarging shelves
  \par found fossils on diff continents where they would fit together,
  including antartica
  \par needed land connection for certain species to travel, i.e. plants seeds
  dont travel far
  \par similar mountain ranges on diff continents. continuation
  \par paleoclimatic evidence:dramatic differnet climates, glacial rock in
  africa
  \par tropical swanps in north
  \subsection{Planet of Shifting plates}
  \par lithospheres, brittle crust
  \par plates move relative one another at slow pace
  \par unequal convection causing movement, large movements of liquid rock
  \subsubsection{Plate boundaries}
  \par all major interactions along boundaries
  \par divergent boundaries, spreading, mid ocean ridge, sea floor spreading
  \par convergent coundaries,descent into mantle, leading edge bent downwared,
  called subduction
  \par transform fault boundaries, no production or destruction, slide past one
  another
  \subsection{Earth's internal structure`}
  \subsubsection{Layers defined by composition}
  \begin{enumerate}
    \item crust:divided into oceanic and continental. basalt in ocean, 7 km
      thick, continent 35-70k. continentla composed of diversity of granite,
      and basalt etc.
    \item mantle: 82\% volume of earth.boundary b/w crust and mantle has
      chemical change. dominant rock periodite, at depth becomes more
      crystalline due to pressure
    \item core:iron-nickel, extemely dense
  \end{enumerate}
  \subsubsection{Physical properties.}
  \par grad. increate in density + temp. but since high pressure, not totally
  liquire
  \begin{enumerate}
    \item Lithosphere and Asthenosphere:crust and transition area, melting
      point
    \item Mesosphere or lower mantle:increased pressure counteracts effects of
      higher temp., capable of gradual flow
    \item Inner and outer core: outer core: liquid, convective flow of
      metallic iron forming mag field. Inner core:behaves like solid even
      though higher temp
  \end{enumerate}
  \subsection{face of earth}
  \par continents and ocean
  \par Continental slope: boundary between continents and deep ocean basins
  \subsubsection{Continents}
  \par young linear mountain belts
  \par shields on stable interiors, large granite crystalline rock, could be 4
  bill years
  \subsubsection{Ocean basins}
  \par linear volcano chains, canyons, large plateaus, extermely thin and deep
  trenches.
  \subsection{Earth as a system}
  \par group of interacting system
  \par system: group of interacting part forming a complex whole.
  \subsubsection{Rock cycle}
  \par recycling of rock.magma formed in earth interior  then migrates to outer
  surface and solidifies to form igneous rocks. on surface undergo weathering
  and become sediment and moves with grabvity or other forces like water
  glacier etc. sediments undergo lithification into sedimentary rock when
  compressed. if buried deep inearth, will be subjected to great pressure and
  turn into metamorphic. Additional pressure or temp turn into magma and cycle
  continues.
\end{document}
