\documentclass[11pt]{amsart}
\usepackage{geometry}                % See geometry.pdf to learn the layout options. There are lots.
\geometry{letterpaper}                   % ... or a4paper or a5paper or ...
%\geometry{landscape}                % Activate for for rotated page geometry
%\usepackage[parfill]{parskip}    % Activate to begin paragraphs with an empty line rather than an indent
\usepackage{graphicx}
\usepackage{amssymb}
\usepackage{epstopdf}
\DeclareGraphicsRule{.tif}{png}{.png}{`convert #1 `dirname #1`/`basename #1 .tif`.png}

\title{Earth121}
\author{Andrew Codispoti}

\begin{document}
\maketitle
\par ESL: Earth science literacy
\par ESLP: Earth science literacy principles Key concepts to know for layman


\section{Introduction}
\subsection{James Hutton}
\par theory of the earth (1788)
\par geologic change is slow large changes require large amount of time
\subsection{principle of uniformitarianism}
\par processes that shaped earth throughout geologic time are the same as those observable today
\subsection{Sir William Logan}
\begin{enumerate}
  \item map coal in ontario
\end{enumerate}
\par gsc, mapping, natural resources
\subsection{J\. Tuzo Wilson}
\par geophysics, transform faults
\subsection{What is geology?}
\begin{enumerate}
  \item study of earth
  \item Earth in greek
  \item \textbf{Physical} material composition of earth
  \item \textbf{historical} development of earth through time(both slow and
    fast)
  \item earth always changing at all levels
  \item outcrops: bedrock on surface {easy for geologists to study}
  \item earth is has many hazardsj
  \item formation and occurence of vital resources
  \item maintaining supply and environmental impact of their extraction and use
\end{enumerate}
\subsection{Catastrophism}
\par earth's landscapes had been shapen by catastrophes. Attempted to explain
shape of earth in fixed time interval that it was a few thousand years old.
4004BCE
\subsection{Uniformitarianism}
\par physical, chemical, and biological laws that operate today always have
been at work
\par wearthering shrunk moutains

\par the present is key to the past
\subsection{The Earth system}
\begin{enumerate}
  \item lithosphere: rock surface and interior
  \item atmosphere: gaseous envelope,weather and climate
  \item hydrosphere: water, liquid, frozen
  \item biosphere: living matter
\end{enumerate}
\par earth powered by sun and heat from internal radioactive decay
\subsection{Rock Cycle}
\par How matter is recycled
\subsection{Earth's layered structure}
\item several different layers with varying physical attributes and chemical compositions
  \subsection{Geologic time}
  \par after radioactivity discovered, can accurately date stuff

  \subsubsection{Relative Dating}
  \par events placed in proper sequence by law of superposition
  \par law of superposition: older stuff on lower layers of sedimentary rock.
  younger on top
  \par principle of fossil sucession: fossil organisms success each other in
  definite order. time span can be determined by fossil content. could
  determine rocks of similar age based on types of fossils
  \subsubsection{Magnitude}
  \par earth 4.6 billion years old so time scale is extermely large
  \par humans a speck in time, long period without life on earth or simple
  organisms
  \subsection{Early evolution of earth}
  \par earth product of remnant from big band
  \par everything H and He
  \par nebular theory:bodies evolved from rotating cloud called solar nebula,
  gas
  \par eventually collapsed and period of contraction where things combine
  including elements
  \par lots of impact on early earth meaning hot, heavy metal sank to bottom
  \par melting formed buoyant masses for early crust
  \par iron core, primitive crust, mantle
  \subsection{Plate Tectonics}
  \par continents fit
  \par continental drift: continents moved across face of planet
  \par paradigm evolved over long period of time
  \subsubsection{continental drift}
  \par wegener ignored first to propose idea
  \par PAngea 200 mill yrs ago
  \par just being able to fit continents wasnt enought evidence
  \par continental shelf:gently sloping shelf platform of continental
  material extending from shore. fit better with this
  \par streams deposit sediment enlarging shelves
  \par found fossils on diff continents where they would fit together,
  including antartica
  \par needed land connection for certain species to travel, i.e. plants seeds
  dont travel far
  \par similar mountain ranges on diff continents. continuation
  \par paleoclimatic evidence:dramatic differnet climates, glacial rock in
  africa
  \par tropical swanps in north
  \subsection{Planet of Shifting plates}
  \par lithospheres, brittle crust
  \par plates move relative one another at slow pace
  \par unequal convection causing movement, large movements of liquid rock
  \subsubsection{Plate boundaries}
  \par all major interactions along boundaries
  \par divergent boundaries, spreading, mid ocean ridge, sea floor spreading
  \par convergent coundaries,descent into mantle, leading edge bent downwared,
  called subduction
  \par transform fault boundaries, no production or destruction, slide past one
  another
  \subsection{Earth's internal structure`}
  \subsubsection{Layers defined by composition}
  \begin{enumerate}
    \item crust:divided into oceanic and continental. basalt in ocean, 7 km
      thick, continent 35-70k. continentla composed of diversity of granite,
      and basalt etc.
    \item mantle: 82\% volume of earth.boundary b/w crust and mantle has
      chemical change. dominant rock periodite, at depth becomes more
      crystalline due to pressure
    \item core:iron-nickel, extemely dense
  \end{enumerate}
  \subsubsection{Physical properties.}
  \par grad. increate in density + temp. but since high pressure, not totally
  liquire
  \begin{enumerate}
    \item Lithosphere and Asthenosphere:crust and transition area, melting
      point
    \item Mesosphere or lower mantle:increased pressure counteracts effects of
      higher temp., capable of gradual flow
    \item Inner and outer core: outer core: liquid, convective flow of
      metallic iron forming mag field. Inner core:behaves like solid even
      though higher temp
  \end{enumerate}
  \subsection{face of earth}
  \par continents and ocean
  \par Continental slope: boundary between continents and deep ocean basins
  \subsubsection{Continents}
  \par young linear mountain belts
  \par shields on stable interiors, large granite crystalline rock, could be 4
  bill years
  \subsubsection{Ocean basins}
  \par linear volcano chains, canyons, large plateaus, extermely thin and deep
  trenches.
  \subsection{Earth as a system}
  \par group of interacting system
  \par system: group of interacting part forming a complex whole.
  \subsubsection{Rock cycle}
  \par recycling of rock.magma formed in earth interior  then migrates to outer
  surface and solidifies to form igneous rocks. on surface undergo weathering
  and become sediment and moves with grabvity or other forces like water
  glacier etc. sediments undergo lithification into sedimentary rock when
  compressed. if buried deep inearth, will be subjected to great pressure and
  turn into metamorphic. Additional pressure or temp turn into magma and cycle
  continues.




  \section{Chapter 12}
  \subsection{Paleomagnetism}
  \par observe historic magnetic field by looking at type of rock and alignemnt
  of poles, as rocks cool, magnetized in parallel to existing magnetic field
  \par fossil magnetism, paleo magnetism
  \subsection{polar wanderin}
  \par look at magnetism of rocks, see that it gradually shifted from somewhere
  in pacific to the canadian arctic. caused by continental drift
  \par ocean floor mapped and drilled and discovered it was new
  \subsection {sea floor spreading}
  \par propose ocean ridges above zones of upwelling and seafloor carried away
  from ridge crest. tensional forces fracture crust and allow magma to intrude
  to create new crust
  \par deep ocean trenches where crust drawn back into earth, older portions
  consumed
  \subsection{geomagnetic reversals}
  \par normal polarity: same magnetism as present magnetic field
  \par reversed polarity: opposite current
  \par alternating stripes of high and low intensity magnetism parallel to
  ridge crest (using magnetometer), high intensity normal polarity, low
  intensity reversed.stripe increase in width as floor spreads
  \par originaly though earth was epxaning but volume and diameter remainded
  constant
  \par tuzo wilson proposed large faults connected global belts into continuous
  network that divides earth outer shell into plates. At ocean ridges, moving
  apart, trenches, moving together, transform faults moving parallel
  \subsection{plate tectonics}
  \par explain observed motion of lithosphere through subduction and sea floor
  spreading
  \par lithosphere surrounds asthenosphere, lithosphere broken into plates
  (change in size and shape),
  \\
  \textbf{Major}
  \begin{enumerate}
    \item north american
    \item south american
    \item pacific african
    \item eurasian
    \item australian-indian
    \item antarctic
  \end{enumerate}
  \\
  \textbf{Minor}
  \begin{enumerate}
    \item carribean
    \item Nazca
    \item Phillippine
    \item Arabian
    \item cocos
    \item scotia
  \end{enumerate}

  \subsection{Divergent plate boundaries}
  \par located along crests of oceanic ridges.as plates move away from
  ridge axis, molten rock wells up to produce slivers of sea floor
  \par magma in fractures form dykes that cool outward -- inward
  \subsubsection{Continental Rifts}
  \par spreading centres developing within continent. How pangea
  separated. caused by opposing tectonic forces pulling lithosphere apart
  \subsection{Convergent plate boundaries}
  \par to accomodate newly formed lithosphere, older potions return to
  mantle on convergent boudnaries and consumed(destructive plate
  boundaries).
  \par leading edge of one plate bends downward (subduction zone). forms
  ocean trench.
  \par when continental plate reaches ocean plate, ocean plate sinks as it
  is denser and heavier while continental plate floats. Some of the
  content on the oceanic plate scraped off on continental
  plate(accretionary wedge)
  \par when descending plate ~ 100km head drives water and voltaile
  components from subducted sediment to overlying mantle. Acts as flux
  inducing partial melting of mantle rocks. magma formed less dense then
  rocks of mantle and rise buoyantly, pool beneath overlying continental
  crust, may give rise to volcanoes (Mount St helens)
  \par two oceanic plates convering cause similar effect, volcanoes build
  upward forming arc shpaed chain of volacnic islands
  \par continents converging create mountains cuz cant sink
  \subsection{transform fault boundaries}
  \par strike-slip faulting, plates move past one another, Tuzo wilson
  \par most transform faults join two segments of mid ocean ridge part of
  prominent linear breaks called fracture zones
  \section{Chapter 2 -- Minerals}
  \subsection{Minerals}
  \begin{enumerate}
    \item minerals: naturally occuring inorganic solids possess and
      orderly internal structure and definite chemical composition.
    \item need to occur naturally, be solid at surface temperature and
      pressure, possess orderly internal structure, definite chemical
      composition, inorganic
    \item rock: solid mass of mineralor mineral like matter that occurs
      naturally
    \item aggregate: combined while properties maintained
  \end{enumerate}
  \subsection{The composition of minerals}
  \begin{enumerate}
    \item every sample of same mineral has same elements
    \item minerals are chemical compounds
    \item ionic compounds have orderly arrangment of oppositely chaged ions
  \end{enumerate}
  \subsection{structure of minerals}
  \begin{enum}
  \item orderly packing of atoms in regularly shaped objects called
    cyrstals
  \item polymorphs:two minerals can have same chemical composition diff physical
    charactersistics.i.e graphite and diamond
  \end{enum}
  \subsection{physical properties}
  \subsubsection{Crystal habit}
  \par external expression of mineral that reflects orderly internal
  arrangment of atoms. When no space restrictions, mineral will form
  individual crystals with well formed cyrstal faces. growth interrupted by
  competition for space. results in intergrown mass lacking habit.
  \par lustre is appearance or quality of ligth reflected form surface of
  mineral, metallic and non metallic, vitreous(glassy), pearly,
  silky,resinous, earthy(dull)
  \par colour not diagnostic
  \par streak is color in a powdered form. rub against porcelain to see
  \par hardness: resistance to abrasion and scratching. rub one of known
  hardness to another to get a numerical value on Mohs scale of relative
  hardness. scale consists of 10 minerals.

  \par cleavage: tendency to break along planes of weak bonding. planes of
  cleavage when minerals break evenly in multiple directions. defined by
  number of planes and angles at which they meet. when cleave, break into
  same geometry
  \par fracture: minerals with chem bonds of same strength can still
  break in distinctive manner. can break into concboidal, splinters,
  fibers, unevern
  \par specific gravity: ratio of weight of a mineral to weight of equal
  volume of water
  \par other properties like taste, elasticity, malleability greasy/soapy
  feel, magnetism, smell, refraction
  \subsection{mineral classes}
  \par most rocks made out of few minerals called rock-forming minerals.
  eight element compose these: O, Si, Al, Fe, Ca, Na, K, Mg
  \subsubsection{Silicates}
  \par silicon and oxygen, most common class
  \par contains one or more common elements needed to produce electrical
  neutrality.
  \par same fundamental building block of silicon-oxygen tetrahedron. 4
  oxygen surroundign silicon. neutralized with addition of cations. whe more
  added, sharing increases and so does sheet structure
  \section{Chapter 3}
  \par bulk of earth's crust. formed when molten rock solidifies and cools.
  \par magma formed by partial melting . once formed magma rises to sruface
  buoyantly because it is less dense than the surrounding rocks. molten
  rock breaking through is a volcanic eruption. lava when on surface
  \par igneous rocks that form when magma solifies at surface alled
  xtrusive or vulcanic. when amgma loses mobility before reaching
  surface, crystallizes and creates plutonic instrusive rock. pluton is body
  of plutonic rock
  \subsection{Generating magma from solid rock}
  \par earths crust and mantle mostly solid rock. magma formed at
  subduciton sones and withing crust. exposed at divergent boudnaries
  \par geothermal gradient: change in temperature as you descend into earth.
  rocks near melting point b ut still solid
  \par additional heat generated by 1)subduction zone friction between
  lithosphere plates 2) crustal rocks heated they descend into mantle during
  subduction. 3) hot mantle rocks can rise and intrude crustal rocks
  \par additional heat generated by 1)subduction zone friction between
  lithosphere plates 2) crustal rocks heated they descend into mantle during
  subduction. 3) hot mantle rocks can rise and intrude crustal rocks. rocks
  near melting point can begin if pressure drops or volatile(fluids, gasses)
  are introduced.
  \subsubsection{role of pressure}
  \par when melting, volume increasing, therefore a decrease in
  pressure will contributre. called decompression melting. rock ascends
  due to convective upwelling
  \subsubsection{role of volatiles}
  \par water and other volatiles decrease melting temp

\end{document}
